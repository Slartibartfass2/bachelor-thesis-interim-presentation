\documentclass[
    aspectratio=169, % default is 43
    8pt % font size, default is 11pt
]{beamer}
\def\templatepath{./fancy-beamer-uulm/}
\usepackage{\templatepath fancybeamer} % use the fancy beamer package
\usepackage{\templatepath fancyuulm} % use the uulm colorpalette
\InputIfFileExists{listings.tex}{%
    \typeout{Loaded listings}%
}{%
    \typeout{Couldn't load listings}%
}%
\LoadLanguage{R}

%\usepackage[ngerman]{babel} % use this line for slides in German

\title[Short Title]{Field-Sensitive Pointer Analysis for Static Dataflow in the R programming language}
\subtitle[Short Subtitle]{Subtitle}
\author[Short Author]{Felix Schlegel}
\date{\today}

% set paths for logos
\setpaths{{\templatepath logos/}}

\fancylogos{sp,uulm} % define logos that are spread evenly across the bottom of the title slide

\begin{document}

\maketitle[example-image][50] % title slide with optional title picture and parameter to move it upwards

\section{First Section}

\subsection{The goal}
\begin{frame}[fragile]{\insertsubsection}
    \begin{fancycolumns}[columns=2]
        \begin{minted}{R}
            person <- list(name = 1, age = 2)
        \end{minted}
        Großes Data Frame Beispiel gesliced
    \nextcolumn
        Großes Data Frame Beispiel genauer gesliced
    \end{fancycolumns}
\end{frame}

\end{document}
